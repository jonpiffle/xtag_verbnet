%
% File acl2017.tex
%
%% Based on the style files for ACL-2015, with some improvements
%%  taken from the NAACL-2016 style
%% Based on the style files for ACL-2014, which were, in turn,
%% based on ACL-2013, ACL-2012, ACL-2011, ACL-2010, ACL-IJCNLP-2009,
%% EACL-2009, IJCNLP-2008...
%% Based on the style files for EACL 2006 by 
%%e.agirre@ehu.es or Sergi.Balari@uab.es
%% and that of ACL 08 by Joakim Nivre and Noah Smith

\documentclass[11pt,a4paper]{article}
\usepackage[hyperref]{acl2017}
\usepackage[export]{adjustbox}
\usepackage{acl2017}
\usepackage{algorithmic}
\usepackage{algorithm}
\usepackage{amsmath}
\usepackage{amssymb}
\usepackage{bm}
\usepackage{booktabs}
\usepackage{float}
\usepackage{graphicx}
\usepackage{hyperref}
\usepackage{latexsym}
\usepackage{multirow}
\usepackage{subcaption}
\usepackage{times}
\usepackage{xcolor}
\usepackage{url}

\usepackage{forest}

%\aclfinalcopy % Uncomment this line for the final submission
%\def\aclpaperid{***} %  Enter the acl Paper ID here

%\setlength\titlebox{5cm}
% You can expand the titlebox if you need extra space
% to show all the authors. Please do not make the titlebox
% smaller than 5cm (the original size); we will check this
% in the camera-ready version and ask you to change it back.

\newcommand\BibTeX{B{\sc ib}\TeX}

% l=height
% l sep = dont know?
% s sep = horizontal spacing
\forestset{.style={for tree={l=2.5em, l sep=0em, s sep=1em, parent anchor=south, child anchor=north,align=center,inner sep=0pt}}}

\title{Instructions for ACL-2017 Proceedings}

\author{First Author \\
  Affiliation / Address line 1 \\
  Affiliation / Address line 2 \\
  Affiliation / Address line 3 \\
  {\tt email@domain} \\\And
  Second Author \\
  Affiliation / Address line 1 \\
  Affiliation / Address line 2 \\
  Affiliation / Address line 3 \\
  {\tt email@domain} \\}

\date{}

\begin{document}
\maketitle
\begin{abstract}
  This document contains the instructions for preparing a camera-ready
  manuscript for the proceedings of ACL-2017. The document itself
  conforms to its own specifications, and is therefore an example of
  what your manuscript should look like. These instructions should be
  used for both papers submitted for review and for final versions of
  accepted papers.  Authors are asked to conform to all the directions
  reported in this document.
\end{abstract}
%!TEX root = acl2017.tex

\section{Introduction and Background}

Lexicalized Tree Adjoining Grammar (LTAG) is a mildly context-sensitive grammar formalism with many desirable linguistic properties \cite{ltag}; however, its adoption has been hindered by the unavailability of a wide-coverage implementation which includes semantic annotations.

The XTAG project is one of the best linguistic resources available. Unlike many available grammars, XTAG ]is a linguistically motivated wide-coverage LTAG for English, 

\begin{figure}
\centering
\Forest{
  [{S: {\em e}}
    [{NP$_{0\downarrow}$}: {\em x} \\ {\em Agent(e,x)}]
    [{VP}
      [{V$\diamond$}
        [chase \\ {\em motion(e)}]]
      [{NP$_{1\downarrow}$}: {\em y} \\ {\em Theme(e,y)}]]]
}
\caption{$\alpha$nx0Vnx1, declarative transitive tree}
\label{fig:declarative}
\end{figure}

\begin{figure}
\centering
\Forest{
  [{S: {\em e}}
    [{NP$_{1\downarrow}$}: {\em y} \\ {\em Theme(e,y)}]
    [{VP}
      [{V$\diamond$}
        [chase \\ {\em motion(e)}]]]]
}
\caption{$\alpha$nx1V, passive without {\em by} phrase transitive tree }
\label{fig:passive}
\end{figure}

\begin{figure}
\centering
\Forest{
  [{NP$_r$: {\em e}}
    [{NP$_{0\downarrow}$}: {\em x} \\ {\em Agent(e,x)}]
    [{VP}
      [{V$\diamond$}
        [chasing \\ {\em motion(e)}]]
      [{NP$_{1\downarrow}$}: {\em y} \\ {\em Theme(e,y)}]]]
}
\caption{$\alpha$Gnx0Vnx1, NP gerund transitive tree}
\label{fig:gerund}
\end{figure}

\begin{figure}
\centering
\Forest{
  [{S$_q$: {\em e}}
    [{NP$_1$}
      [{what}]]
    [{S$_r$}
      [{NP$_{0\downarrow}$}: {\em x} \\ {\em Agent(e,x)}]
      [{VP}
        [{V$\diamond$}
          [chase \\ {\em motion(e)}]]
      [{NP}
        [{$\epsilon$}]]]]]
}
\caption{$\alpha$W1nx0Vnx1, wh-extraction of NP$_1$ transitive tree}
\label{fig:whmovement}
\end{figure}

\begin{figure}
\centering
\Forest{
  [{NP$_r$: {\em y}}
    [{NP$_f\star$ \\ {\em Theme(e,y)}}]
    [{S$_p$: $e$}
      [{NP$_w$}
        [{which}]]
      [{S$_r$}
        [{NP$_{0\downarrow}$}: {\em x} \\ {\em Agent(e,x)}]
        [{VP}
          [{V$\diamond$}
            [chase \\ {\em motion(e)}]]
          [{NP$_1$}
            [{$\epsilon$}]]]]]]
}
\caption{$\beta$N1nx0Vnx1, object relative clause extraction with overt wh-NP transitive tree}
\label{fig:relclause}
\end{figure}

\begin{figure}
\centering
\Forest{
  [{S$_r$: {\em e}}
    [{NP$_0$}
      [{PRO}]]
    [{VP}
      [{V$\diamond$}
        [chase \\ {\em motion(e)}]]
      [{NP$_{1\downarrow}$}: {\em y} \\ {\em Theme(e,y)}]]]
}
\caption{$\alpha$nx0Vnx1-PRO, PRO-subject transitive tree}
\label{fig:pro}
\end{figure}


\input{background}
\input{approach}
\input{evaluation}
\input{conclusion}

% include your own bib file like this:
%\bibliographystyle{acl}
%\bibliography{acl2017}
\bibliography{acl2017}
\bibliographystyle{acl_natbib}

\end{document}
